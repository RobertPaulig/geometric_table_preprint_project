\documentclass[11pt]{article}

\usepackage[margin=1in]{geometry}
\usepackage{amsmath, amssymb}
\usepackage{graphicx}
\usepackage{hyperref}

\title{Wave Atlas: Geometric Table Fronts (M1)}
\author{Working note}
\date{\today}

\begin{document}
\maketitle

\section{Definitions}
Let $M(n,k)=\mathbf{1}_{\{k\mid n\}}$ be the occupancy matrix of the Geometric Table,
and $V(n,k)=n/k$ on cells where $k\mid n$ (empty otherwise). For a fixed diagonal
center $N$, define the diagonal profile
\[
D_N(k)=N/k-1,\qquad k\le K,\ k\mid N.
\]
We visualize three ``wave'' views: occupancy, log-values, and diagonal hits.

\section{Occupancy and values}
\begin{figure}[h]
\centering
\includegraphics[width=0.9\linewidth]{../out/wave_atlas/divisibility_occ_N3000_K120.png}
\caption{Occupancy heatmap $M(n,k)$ for $N_{\max}=3000$, $K=120$.}
\end{figure}

\begin{figure}[h]
\centering
\includegraphics[width=0.9\linewidth]{../out/wave_atlas/divisibility_val_log_N3000_K120.png}
\caption{Log-values heatmap $\log(1+V(n,k))$ on occupied cells.}
\end{figure}

\paragraph{Scroll.}
The scroll animation is saved as
\texttt{out/wave\_atlas/divisibility\_occ\_scroll\_N3000\_K120\_H220\_step60.gif}.

\section{Diagonal profiles}
\begin{figure}[h]
\centering
\includegraphics[width=0.49\linewidth]{../out/wave_atlas/diagonal_hits_N60_K120.png}
\includegraphics[width=0.49\linewidth]{../out/wave_atlas/diagonal_hits_N420_K120.png}
\caption{Diagonal profiles for $N=60$ and $N=420$.}
\end{figure}

\begin{figure}[h]
\centering
\includegraphics[width=0.49\linewidth]{../out/wave_atlas/diagonal_hits_N2520_K120.png}
\includegraphics[width=0.49\linewidth]{../out/wave_atlas/diagonal_hits_N27720_K120.png}
\caption{Diagonal profiles for $N=2520$ and $N=27720$.}
\end{figure}

\section{Wave Metrics (M2)}
Wave patterns can be decomposed into three interacting structures: periodic combs
(column densities), rays $n=qk$ (counts by $q$), and diagonal impulses (divisor hits).

\begin{figure}[h]
\centering
\includegraphics[width=0.9\linewidth]{../out/wave_atlas/metrics/col_density_heatmap.png}
\caption{Column density heatmap across sliding windows (x=$k$, y=window start). Clear periodic combs appear at small $k$.}
\end{figure}

\begin{figure}[h]
\centering
\includegraphics[width=0.9\linewidth]{../out/wave_atlas/metrics/q_top_bars_win1.png}
\caption{Top $q$ counts in the first window (rays $n=qk$). Dominant $q$ values trace the most frequent rays.}
\end{figure}

\begin{figure}[h]
\centering
\includegraphics[width=0.9\linewidth]{../out/wave_atlas/metrics/diag_hits_raster_N2520_K120.png}
\caption{Diagonal hits raster for $N=2520$ (hit/non-hit over $k$). Impulses align with dense divisor blocks.}
\end{figure}

\section{Baselines and scaling (M3)}
Two deterministic baselines serve as controls. First, the expected occupancy density
over a window is $H_K/K$ where $H_K=\sum_{k\le K}1/k$, and the measured densities
match this baseline closely. Second, column densities follow the $1/k$ law.

\begin{figure}[h]
\centering
\includegraphics[width=0.9\linewidth]{../out/wave_atlas/metrics/col_density_vs_1_over_k.png}
\caption{Mean column density versus the $1/k$ baseline. Deviations are small.}
\end{figure}

\begin{table}[h]
\centering
\begin{tabular}{lccc}
\hline
$K$ & $\overline{\text{occ}}$ & $H_K/K$ & $|\Delta|$ \\
\hline
60  & 0.07764 & 0.077998 & 0.000354 \\
120 & 0.04452 & 0.044741 & 0.000224 \\
240 & 0.02510 & 0.025250 & 0.000145 \\
\hline
\end{tabular}
\caption{Scaling check: measured occupancy density matches $H_K/K$.}
\end{table}

\paragraph{Ray-count identity.}
For each window and $q$, the observed ray count equals the exact formula
$\max(0, \min(K,\lfloor (a+H-1)/q\rfloor) - \lceil a/q\rceil + 1)$, confirming
that the $q$-structure is deterministic geometry rather than noise.

\section{M4 Wheel Overlay}
Wheel centers $N=t\cdot L_m$ fix a dense diagonal backbone from $L_m$ while $t$ modulates
local smoothness. We overlay wave-style features on these centers and compare twin
vs non-twin cases.

\begin{figure}[h]
\centering
\includegraphics[width=0.9\linewidth]{../out/wave_atlas/m4/m4_runlen_hist.png}
\caption{Diagonal run length ($K=120$) for wheel centers: twin vs non-twin.}
\end{figure}

\begin{figure}[h]
\centering
\includegraphics[width=0.9\linewidth]{../out/wave_atlas/m4/m4_tdiv_hist.png}
\caption{$t$-small-divisor counts for wheel centers: twin vs non-twin.}
\end{figure}

\begin{figure}[h]
\centering
\includegraphics[width=0.9\linewidth]{../out/wave_atlas/m4/m4_scatter_runlen_vs_tdiv.png}
\caption{Scatter of diagonal run length vs $t$-divisor count (twins highlighted).}
\end{figure}

\begin{table}[h]
\centering
\begin{tabular}{lcc}
\hline
Feature & Twin (mean/median) & Non-twin (mean/median) \\
\hline
diag\_run\_len\_K & 12.33 / 12 & 12.27 / 12 \\
t\_small\_div\_count & 5.40 / 4 & 5.37 / 4 \\
\hline
\end{tabular}
\caption{Summary stats for wheel overlay features at $m=12$, $K=120$.}
\end{table}

\section{M5 Twin signal on the $t$-axis}
We analyze the twin indicator $x_t$ along the $t$-axis for fixed wheel $B=L_m$,
looking for periodic structure beyond the DC component. We report FFT power,
autocorrelation, and modular lifts across residue classes.

\begin{figure}[h]
\centering
\includegraphics[width=0.9\linewidth]{../out/wave_atlas/m5/fft_power.png}
\caption{FFT power of the centered twin indicator $x_t-\bar{x}$.}
\end{figure}

\begin{figure}[h]
\centering
\includegraphics[width=0.9\linewidth]{../out/wave_atlas/m5/autocorr.png}
\caption{Autocorrelation of the twin indicator (lags up to 5000).}
\end{figure}

\begin{figure}[h]
\centering
\includegraphics[width=0.9\linewidth]{../out/wave_atlas/m5/mod_m420_lift.png}
\caption{Residue-class lift for $m=420$ (relative to global twin rate).}
\end{figure}

\begin{table}[h]
\centering
\begin{tabular}{lccc}
\hline
Peak & $f_{\mathrm{idx}}$ & period $\approx T/f_{\mathrm{idx}}$ & perm-$p$ \\
\hline
1 & 30769 & 6.50 & 0.000 \\
2 & 92308 & 2.17 & 0.000 \\
3 & 82353 & 2.43 & 0.000 \\
4 & 94737 & 2.11 & 0.000 \\
5 & 76923 & 2.60 & 0.000 \\
\hline
\end{tabular}
\caption{Top FFT peaks (from $T=200000$). Periods are approximate; perm-$p$ uses shuffled controls.}
\end{table}

\paragraph{Autocorr and modular lifts.}
Top autocorrelation lags cluster around $\ell\in\{663, 741, \dots\}$ with small magnitude
($\approx 0.02$), and modular lifts show mild non-uniformity (e.g.\ $m=420$).
These signals are structured but not strongly separating at this resolution.

\end{document}
