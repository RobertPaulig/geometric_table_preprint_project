\documentclass[11pt]{article}

\usepackage[margin=1in]{geometry}
\usepackage{amsmath, amssymb}
\usepackage{graphicx}
\usepackage{hyperref}

\title{Wave Atlas: Geometric Table Fronts (M1)}
\author{Working note}
\date{Wave Atlas v1.1}

\begin{document}
\maketitle

\section{Definitions}
Let $M(n,k)=\mathbf{1}_{\{k\mid n\}}$ be the occupancy matrix of the Geometric Table,
and $V(n,k)=n/k$ on cells where $k\mid n$ (empty otherwise). For a fixed diagonal
center $N$, define the diagonal profile
\[
D_N(k)=N/k-1,\qquad k\le K,\ k\mid N.
\]
We visualize three ``wave'' views: occupancy, log-values, and diagonal hits.

\section{Occupancy and values}
\begin{figure}[h]
\centering
\includegraphics[width=0.9\linewidth]{../out/wave_atlas/divisibility_occ_N3000_K120.png}
\caption{Occupancy heatmap $M(n,k)$ for $N_{\max}=3000$, $K=120$.}
\end{figure}

\begin{figure}[h]
\centering
\includegraphics[width=0.9\linewidth]{../out/wave_atlas/divisibility_val_log_N3000_K120.png}
\caption{Log-values heatmap $\log(1+V(n,k))$ on occupied cells.}
\end{figure}

\paragraph{Scroll.}
The scroll animation is saved as
\texttt{out/wave\_atlas/divisibility\_occ\_scroll\_N3000\_K120\_H220\_step60.gif}.

\section{Diagonal profiles}
\begin{figure}[h]
\centering
\includegraphics[width=0.49\linewidth]{../out/wave_atlas/diagonal_hits_N60_K120.png}
\includegraphics[width=0.49\linewidth]{../out/wave_atlas/diagonal_hits_N420_K120.png}
\caption{Diagonal profiles for $N=60$ and $N=420$.}
\end{figure}

\begin{figure}[h]
\centering
\includegraphics[width=0.49\linewidth]{../out/wave_atlas/diagonal_hits_N2520_K120.png}
\includegraphics[width=0.49\linewidth]{../out/wave_atlas/diagonal_hits_N27720_K120.png}
\caption{Diagonal profiles for $N=2520$ and $N=27720$.}
\end{figure}

\section{Wave Metrics (M2)}
Wave patterns can be decomposed into three interacting structures: periodic combs
(column densities), rays $n=qk$ (counts by $q$), and diagonal impulses (divisor hits).

\begin{figure}[h]
\centering
\includegraphics[width=0.9\linewidth]{../out/wave_atlas/metrics/col_density_heatmap.png}
\caption{Column density heatmap across sliding windows (x=$k$, y=window start). Clear periodic combs appear at small $k$.}
\end{figure}

\begin{figure}[h]
\centering
\includegraphics[width=0.9\linewidth]{../out/wave_atlas/metrics/q_top_bars_win1.png}
\caption{Top $q$ counts in the first window (rays $n=qk$). Dominant $q$ values trace the most frequent rays.}
\end{figure}

\begin{figure}[h]
\centering
\includegraphics[width=0.9\linewidth]{../out/wave_atlas/metrics/diag_hits_raster_N2520_K120.png}
\caption{Diagonal hits raster for $N=2520$ (hit/non-hit over $k$). Impulses align with dense divisor blocks.}
\end{figure}

\section{Baselines and scaling (M3)}
Two deterministic baselines serve as controls. First, the expected occupancy density
over a window is $H_K/K$ where $H_K=\sum_{k\le K}1/k$, and the measured densities
match this baseline closely. Second, column densities follow the $1/k$ law.

\begin{figure}[h]
\centering
\includegraphics[width=0.9\linewidth]{../out/wave_atlas/metrics/col_density_vs_1_over_k.png}
\caption{Mean column density versus the $1/k$ baseline. Deviations are small.}
\end{figure}

\begin{table}[h]
\centering
\begin{tabular}{lccc}
\hline
$K$ & $\overline{\text{occ}}$ & $H_K/K$ & $|\Delta|$ \\
\hline
60  & 0.07764 & 0.077998 & 0.000354 \\
120 & 0.04452 & 0.044741 & 0.000224 \\
240 & 0.02510 & 0.025250 & 0.000145 \\
\hline
\end{tabular}
\caption{Scaling check: measured occupancy density matches $H_K/K$.}
\end{table}

\paragraph{Ray-count identity.}
For each window and $q$, the observed ray count equals the exact formula
$\max(0, \min(K,\lfloor (a+H-1)/q\rfloor) - \lceil a/q\rceil + 1)$, confirming
that the $q$-structure is deterministic geometry rather than noise.

\section{M4 Wheel Overlay}
Wheel centers $N=t\cdot L_m$ fix a dense diagonal backbone from $L_m$ while $t$ modulates
local smoothness. We overlay wave-style features on these centers and compare twin
vs non-twin cases.

\begin{figure}[h]
\centering
\includegraphics[width=0.9\linewidth]{../out/wave_atlas/m4/m4_runlen_hist.png}
\caption{Diagonal run length ($K=120$) for wheel centers: twin vs non-twin.}
\end{figure}

\begin{figure}[h]
\centering
\includegraphics[width=0.9\linewidth]{../out/wave_atlas/m4/m4_tdiv_hist.png}
\caption{$t$-small-divisor counts for wheel centers: twin vs non-twin.}
\end{figure}

\begin{figure}[h]
\centering
\includegraphics[width=0.9\linewidth]{../out/wave_atlas/m4/m4_scatter_runlen_vs_tdiv.png}
\caption{Scatter of diagonal run length vs $t$-divisor count (twins highlighted).}
\end{figure}

\begin{table}[h]
\centering
\begin{tabular}{lcc}
\hline
Feature & Twin (mean/median) & Non-twin (mean/median) \\
\hline
diag\_run\_len\_K & 12.33 / 12 & 12.27 / 12 \\
t\_small\_div\_count & 5.40 / 4 & 5.37 / 4 \\
\hline
\end{tabular}
\caption{Summary stats for wheel overlay features at $m=12$, $K=120$.}
\end{table}

\section{M5 Twin signal on the $t$-axis}
We analyze the twin indicator $x_t$ along the $t$-axis for fixed wheel $B=L_m$,
looking for periodic structure beyond the DC component. We report FFT power,
autocorrelation, and modular lifts across residue classes.

\begin{figure}[h]
\centering
\includegraphics[width=0.9\linewidth]{../out/wave_atlas/m5/fft_power.png}
\caption{FFT power of the centered twin indicator $x_t-\bar{x}$.}
\end{figure}

\begin{figure}[h]
\centering
\includegraphics[width=0.9\linewidth]{../out/wave_atlas/m5/autocorr.png}
\caption{Autocorrelation of the twin indicator (lags up to 5000).}
\end{figure}

\begin{figure}[h]
\centering
\includegraphics[width=0.9\linewidth]{../out/wave_atlas/m5/mod_m420_lift.png}
\caption{Residue-class lift for $m=420$ (relative to global twin rate).}
\end{figure}

\begin{table}[h]
\centering
\begin{tabular}{lccc}
\hline
Peak & $f_{\mathrm{idx}}$ & period $\approx T/f_{\mathrm{idx}}$ & perm-$p$ \\
\hline
1 & 30769 & 6.50 & 0.000 \\
2 & 92308 & 2.17 & 0.000 \\
3 & 82353 & 2.43 & 0.000 \\
4 & 94737 & 2.11 & 0.000 \\
5 & 76923 & 2.60 & 0.000 \\
\hline
\end{tabular}
\caption{Top FFT peaks (from $T=200000$). Periods are approximate; perm-$p$ uses shuffled controls.}
\end{table}

\paragraph{Autocorr and modular lifts.}
Top autocorrelation lags cluster around $\ell\in\{663, 741, \dots\}$ with small magnitude
($\approx 0.02$), and modular lifts show mild non-uniformity (e.g.\ $m=420$).
These signals are structured but not strongly separating at this resolution.

\section{M6 Wheel-shift control (m=10 vs m=12)}
We repeat the $t$-axis analysis with $m=10$ (wheel $B=L_{10}=2520$) at comparable $N_{\max}$
to test whether the dominant signatures shift to the smallest unfiltered prime (11).
Raw CSV for $m=10$ is stored compressed in \texttt{out/wheel\_scan\_m10\_t2200000.csv.gz}.

\begin{table}[h]
\centering
\begin{tabular}{lcc}
\hline
Wheel & top-lags divisible by $p$ & top-moduli divisible by $p$ \\
\hline
$m=10$ ($p=11$) & 18/20 & 16/20 \\
$m=12$ ($p=13$) & 18/20 & 17/20 \\
\hline
\end{tabular}
\caption{Divisibility of top autocorr lags and top chi2 moduli by the first unfiltered prime.}
\end{table}

\begin{table}[h]
\centering
\begin{tabular}{lcc}
\hline
Wheel & peak periods (approx.) & comment \\
\hline
$m=10$ & 11.0, 2.2, 2.75, 4.25 & harmonics of 11 \\
$m=12$ & 6.50, 2.17, 2.43, 2.60 & harmonics of 13/2 \\
\hline
\end{tabular}
\caption{FFT peak periods shift with the first unfiltered prime.}
\end{table}

\section{M7 Detrend and conditioning}
We detrend the twin indicator with a rolling mean and then condition on the
forbidden classes mod $p_0$ (removing the dominant wheel signature). Segment
analysis shows mild nonstationarity, so detrending is applied before spectral
comparisons. For m=10, conditional modulus scans are limited to $m\le 210$
for compute budget.

\begin{figure}[h]
\centering
\includegraphics[width=0.9\linewidth]{../out/wave_atlas/m7/m12/segment_rates.png}
\caption{Segmented twin rate (m=12, segment length 20k). Rates drift but remain stable at scale.}
\end{figure}

\begin{figure}[h]
\centering
\includegraphics[width=0.9\linewidth]{../out/wave_atlas/m7/m12/detrend_fft_power_detrended.png}
\caption{Detrended FFT power (m=12). Dominant peaks reduce in amplitude after detrending.}
\end{figure}

\begin{figure}[h]
\centering
\includegraphics[width=0.9\linewidth]{../out/wave_atlas/m7/m12/cond_p0_fft_power.png}
\caption{Conditional FFT after removing $p_0=13$ forbidden classes (m=12).}
\end{figure}

\begin{figure}[h]
\centering
\includegraphics[width=0.98\linewidth]{../out/wave_atlas/m7/cond_fft_overlay_m10_m12.png}
\caption{Detrended vs conditioned spectra for m=10 and m=12. Conditioning removes the dominant $p_0$ ridge and leaves weaker structure.}
\end{figure}

\begin{table}[h]
\centering
\begin{tabular}{lcccc}
\hline
Wheel & rank & period & best fit & rel.\ err. \\
\hline
$m=10$ & 1 & 10.64 & $23/(2\cdot 1)$ & 0.081 \\
$m=10$ & 2 & 5.32 & $17/3$ & 0.066 \\
$m=10$ & 3 & 3.48 & $17/5$ & 0.022 \\
$m=12$ & 1 & 4.79 & $29/(2\cdot 3)$ & 0.008 \\
$m=12$ & 2 & 2.05 & $29/(2\cdot 7)$ & 0.008 \\
$m=12$ & 3 & 2.27 & $23/(2\cdot 5)$ & 0.012 \\
\hline
\end{tabular}
\caption{Top periods after conditioning ($p_0$ removed) with nearest harmonic fits to the next primes.}
\end{table}

\begin{table}[h]
\centering
\begin{tabular}{lcc}
\hline
Wheel & top-lags divisible by $p_0$ & top-moduli divisible by $p_0$ \\
\hline
$m=10$ ($p_0=11$) & 18/20 & 16/20 \\
$m=12$ ($p_0=13$) & 18/20 & 17/20 \\
\hline
\end{tabular}
\caption{Persistence of $p_0$ signature in M7; conditioning reduces but does not eliminate structure.}
\end{table}

\section{M8 Sequential conditioning}
We apply conditioning in layers: remove $p_0$, then $p_1$, then $p_2$, and track
the remaining spectral structure.

\begin{figure}[h]
\centering
\includegraphics[width=0.9\linewidth]{../out/wave_atlas/m8/m10/m8_fft_layers.png}
\caption{Sequential conditioning layers for m=10 (p0=11, p1=13, p2=17).}
\end{figure}

\begin{figure}[h]
\centering
\includegraphics[width=0.9\linewidth]{../out/wave_atlas/m8/m12/m8_fft_layers.png}
\caption{Sequential conditioning layers for m=12 (p0=13, p1=17, p2=19).}
\end{figure}

\begin{table}[h]
\centering
\begin{tabular}{lccc}
\hline
Wheel & layer & top period & fit \\
\hline
$m=10$ & 1 & 6.50 & $13/(2\cdot 1)$ \\
$m=10$ & 2 & 5.67 & $17/3$ \\
$m=10$ & 3 & 2.71 & $19/7$ \\
$m=12$ & 1 & 5.67 & $17/3$ \\
$m=12$ & 2 & 2.71 & $19/7$ \\
$m=12$ & 3 & 2.04 & $41/(2\cdot 10)$ \\
\hline
\end{tabular}
\caption{Top periods by layer after sequential conditioning (M8).}
\end{table}

Sequential conditioning removes successive modular combs induced by the first
non-wheel primes. Peak energy decays by orders of magnitude across layers, and
the remaining peaks align with the next prime layer rather than an independent
periodic mechanism.

\section{M9 Bridge to GT geometry}
We sample centers $c=B\cdot t$ from the wheel axis and compute GT core metrics
in the same $K$-window as the wave atlas. Core density varies by $t \bmod p_0$,
and conditioning on successive layers reshapes the distribution of GT metrics.
After controlling for layers, the global GC gap remains non-separating, while
local twin-row features stay strongly separated.

\begin{figure}[h]
\centering
\includegraphics[width=0.9\linewidth]{../out/wave_atlas/m9/m12/m9_core_edges_by_mod_p0.png}
\caption{Mean core\_edges by residue class $t \bmod p_0$ (m=12, $p_0=13$).}
\end{figure}

\begin{figure}[h]
\centering
\includegraphics[width=0.9\linewidth]{../out/wave_atlas/m9/m12/m9_layers_box_core_edges.png}
\caption{core\_edges distributions under layer conditioning (m=12).}
\end{figure}

\begin{figure}[h]
\centering
\includegraphics[width=0.9\linewidth]{../out/wave_atlas/m9/m12/m9_twin_non_twin_isolates.png}
\caption{Twin vs non-twin for local twin-row isolates (m=12).}
\end{figure}

\paragraph{M9b (wheel-lattice rows).}
To avoid degeneration at large centers, we compute core metrics on a wheel
row-lattice $n=B\cdot(t-r..t+r)$ with fixed $K$. This restores variability and
reveals residue-class structure.
Increasing $K$ from 120 to 240 raises variability (unique core\_edges 13$\to$19)
while preserving the same residue-class effects, indicating this bridge is not
a fragile $K$ artifact.

\begin{figure}[h]
\centering
\includegraphics[width=0.9\linewidth]{../out/wave_atlas/m9b/m12/K120/m9_core_edges_by_mod_p0.png}
\caption{Wheel-lattice core\_edges by $t \bmod p_0$ (m=12, K=120).}
\end{figure}

\begin{figure}[h]
\centering
\includegraphics[width=0.9\linewidth]{../out/wave_atlas/m9b/m12/K120/m9_layers_box_core_edges.png}
\caption{Wheel-lattice layer conditioning effects on core\_edges (m=12, K=120).}
\end{figure}

\begin{figure}[h]
\centering
\includegraphics[width=0.9\linewidth]{../out/wave_atlas/m9b/m12/K240/m9_core_edges_by_mod_p0.png}
\caption{Wheel-lattice core\_edges by $t \bmod p_0$ (m=12, K=240).}
\end{figure}

\begin{figure}[h]
\centering
\includegraphics[width=0.9\linewidth]{../out/wave_atlas/m9b/m12/K240/m9_layers_box_core_edges.png}
\caption{Wheel-lattice layer conditioning effects on core\_edges (m=12, K=240).}
\end{figure}

\appendix
\section{Roadmap / Future Work}
Planned milestones are listed in \texttt{docs/ROADMAP.md}. We summarize the
next steps as measurable deliverables (no claims of proof).

\begin{tabular}{lp{0.72\linewidth}}
\hline
M10 & Repro/CI polish: unified build command and stable PDF checksum. \\
M11 & Multi-wheel survey (m=8..14): energy decay by layers, p0 signatures. \\
M12 & Residual-as-process after 3 layers: gaps, ACF, goodness-of-fit. \\
M13 & GT wheel-lattice metrics: identify graph features sensitive to layers. \\
M14 & Candidate generator: layer-based acceleration and survival curves. \\
M15 & Mersenne wave atlas (optional): ord\_q(2) structure and conditioning. \\
\hline
\end{tabular}

\section{M11 Multi-wheel survey}
We run the sequential conditioning pipeline for $m=8..14$ with fixed $t_{\max}=200000$.
The dominant layer follows the first prime $p_0>m$, and energy decays by orders
of magnitude across layers.

\begin{table}[h]
\centering
\scriptsize
\begin{tabular}{r r r r r r r r}
\hline
m & B & p0 & p1 & p2 & E1 & E2 & E3 \\
\hline
8 & 840 & 11 & 13 & 17 & 6.31e+05 & 5.89e+03 & 1.43e+02 \\
9 & 2520 & 11 & 13 & 17 & 8.55e+05 & 4.53e+03 & 1.47e+02 \\
10 & 2520 & 11 & 13 & 17 & 8.55e+05 & 4.53e+03 & 1.47e+02 \\
11 & 27720 & 13 & 17 & 19 & 2.91e+05 & 1.95e+03 & 7.21e+01 \\
12 & 27720 & 13 & 17 & 19 & 2.91e+05 & 1.95e+03 & 7.21e+01 \\
13 & 360360 & 17 & 19 & 23 & 1.19e+05 & 1.06e+03 & 3.64e+01 \\
14 & 360360 & 17 & 19 & 23 & 1.19e+05 & 1.06e+03 & 3.64e+01 \\
\hline
\end{tabular}
\caption{Multi-wheel summary (m=8..14). E1..E3 are top-peak energies after layers 1..3.}
\end{table}

\begin{figure}[h]
\centering
\includegraphics[width=0.9\linewidth]{../out/wave_atlas/m11/m11_energy_decay.png}
\caption{Energy decay by layer across wheels (m=8..14).}
\end{figure}

\section{M12 Residual-as-process}
After removing three sieve layers, we study the residual event process on the
allowed axis. The residual looks weakly structured, closer to noise than to a
strong periodic signal.
An inhomogeneous null model with local intensity $\hat p(i)$ explains part of
the overdispersion, but observed dispersion remains above the null band.

\begin{figure}[h]
\centering
\includegraphics[width=0.9\linewidth]{../out/wave_atlas/m12/m12_compare_dispersion.png}
\caption{Dispersion index (Var/Mean) of residual counts across m=10,12,14.}
\end{figure}

\begin{figure}[h]
\centering
\includegraphics[width=0.9\linewidth]{../out/wave_atlas/m12/null/m12_dispersion_null_compare.png}
\caption{Observed dispersion vs inhomogeneous null (mean $\pm$ 90\% band).}
\end{figure}

\begin{figure}[h]
\centering
\includegraphics[width=0.9\linewidth]{../out/wave_atlas/m12/m12/residual_gaps.png}
\caption{Residual gap distribution for m=12 after three layers.}
\end{figure}

\section{M13 GT wheel-lattice metrics}
We evaluate additional graph metrics on wheel-lattice rows ($B=L_{12}$) to find
features that respond to sieve layers beyond core\_edges/gap/entropy.
We scan $(K,\varepsilon)\in\{120,240\}\times\{10^{-12},10^{-9},10^{-6}\}$ and
summarize sensitivity by residue class and layer conditioning.

\begin{figure}[h]
\centering
\includegraphics[width=0.9\linewidth]{../out/wave_atlas/m13/m13_best_metric_mean_by_mod_p0.png}
\caption{Best GT metric mean by $t \bmod p_0$ (M13).}
\end{figure}

\begin{figure}[h]
\centering
\includegraphics[width=0.9\linewidth]{../out/wave_atlas/m13/m13_best_metric_by_layer_box.png}
\caption{Best GT metric by layer conditioning (M13).}
\end{figure}

\begin{table}[h]
\centering
\begin{tabular}{lcc}\hline
metric & K & eps \\\hline
triangle\_count & 240 & 1e-12 \\
isolated\_nodes & 240 & 1e-12 \\
\hline\end{tabular}

\caption{Top-2 GT metrics by sensitivity score (M13).}
\end{table}

\section{M14 Candidate generator (wave-sieve accelerator)}
We construct candidate centers by enforcing layered sieve constraints on $t$
and measure the practical survival rate and generation throughput as layers
increase. This provides a direct view of how modular filtering accelerates
the search space.

\begin{figure}[h]
\centering
\includegraphics[width=0.9\linewidth]{../out/wave_atlas/m14/m14_survival_vs_layers.png}
\caption{Survival rate versus number of sieve layers (M14).}
\end{figure}

\begin{figure}[h]
\centering
\includegraphics[width=0.9\linewidth]{../out/wave_atlas/m14/m14_throughput_vs_layers.png}
\caption{Candidate throughput versus number of sieve layers (M14).}
\end{figure}

\section{M14b Segmented layers (accelerator scaling)}
We avoid the exploding modulus $L=\prod p$ by applying the same layer filters
in fixed $t$ segments. This lets us extend to dozens of layers while keeping
runtime stable, and directly measure acceleration at large depth.

\begin{figure}[h]
\centering
\includegraphics[width=0.9\linewidth]{../out/wave_atlas/m14b/m14b_survival_vs_layers.png}
\caption{Segmented survival rate versus number of layers (M14b).}
\end{figure}

\begin{figure}[h]
\centering
\includegraphics[width=0.9\linewidth]{../out/wave_atlas/m14b/m14b_throughput_vs_layers.png}
\caption{Segmented throughput versus number of layers (M14b).}
\end{figure}

\end{document}
