% geometric_table_snowflakes_preprint.tex
% Title: Geometric Table Snowflakes: Divisibility Geometry, Primitive Factor Graphs, and Spectral Fingerprints of Integers
% Compile: latexmk -lualatex -interaction=nonstopmode geometric_table_snowflakes_preprint.tex
\documentclass[11pt]{article}

% ----- Packages -----
\usepackage{iftex}
\usepackage[margin=1in]{geometry}
\usepackage{amsmath, amssymb, amsthm, mathtools}
\usepackage{bm}
\usepackage{microtype}
\usepackage{graphicx}
\usepackage{booktabs}
\usepackage{enumitem}
\usepackage{hyperref}
\usepackage{xcolor}
\usepackage{algorithm}
\usepackage{algpseudocode}

% ----- Fonts (LuaLaTeX/XeLaTeX) -----
\ifPDFTeX
  % pdfLaTeX fallback (kept for compatibility)
\else
  \usepackage{fontspec}
  \setmainfont{Latin Modern Roman}
  \setsansfont{Latin Modern Sans}
  \setmonofont{Latin Modern Mono}
\fi

% ----- Hyperref -----
\hypersetup{
  colorlinks=true,
  linkcolor=blue,
  citecolor=blue,
  urlcolor=blue
}

% ----- Theorem environments -----
\newtheorem{theorem}{Theorem}[section]
\newtheorem{lemma}[theorem]{Lemma}
\newtheorem{proposition}[theorem]{Proposition}
\newtheorem{corollary}[theorem]{Corollary}
\theoremstyle{definition}
\newtheorem{definition}[theorem]{Definition}
\theoremstyle{remark}
\newtheorem{remark}[theorem]{Remark}

% ----- Macros -----
\newcommand{\GT}{\mathrm{GT}}
\newcommand{\lcm}{\mathrm{lcm}}
\renewcommand{\gcd}{\mathrm{gcd}}
\newcommand{\1}{\mathbf{1}}
\newcommand{\N}{\mathbb{N}}
\newcommand{\Z}{\mathbb{Z}}
\newcommand{\R}{\mathbb{R}}
\newcommand{\Q}{\mathbb{Q}}

\title{\bfseries Geometric Table Snowflakes: Divisibility Geometry, Primitive Factor Graphs, and Spectral Fingerprints of Integers}
\author{Robert Paulig\\
\small (working draft; contact: \texttt{btc.paulig@gmail.com})}
\date{\today}

\begin{document}
\maketitle

\begin{abstract}
We introduce the \emph{Geometric Table}: a sparse divisibility matrix whose non-empty cells encode factor pairs and induce a geometry of rational rays on the integer lattice. From local windows of the table we build \emph{primitive factor graphs} and study their Laplacian spectra as reproducible fingerprints of integer structure. We connect (i) diagonal ``ray algebra'' around a node $N$ to the divisor sets $d\pm 1$; (ii) a bipartite graph model (numbers $\leftrightarrow$ rays) filtered by primitive factor pairs; and (iii) a reproducible computational protocol (``evidence packs'') for spectral comparisons of classes of centers.
\end{abstract}

\tableofcontents

\section{Introduction}
This draft studies a single object --- the Geometric Table --- from three synchronized perspectives: (a) geometry of rays, (b) graphs induced by factor pairs, and (c) spectral invariants. The immediate goal is a reproducible computational framework; the long-term goal is to test whether spectral signatures correlate with prime-structure events (e.g., centers of twin primes).

\paragraph{Snowflakes.}
We call an even integer $N$ a \emph{full snowflake} if $N-1$ and $N+1$ are prime (equivalently, $N$ is the midpoint of a twin-prime pair). Proving infinitely many full snowflakes is equivalent to the Twin Prime Conjecture; nevertheless, our framework supports \emph{partial} snowflakes (sieve-level conditions) that are provably infinite via CRT and may still exhibit distinctive spectral structure.

\section{Geometric Table}
\subsection{Divisibility matrix and support}
Fix window parameters $(N_{\max},K)$. Define
\begin{equation}
T_{N_{\max},K}(n,k)=
\begin{cases}
\frac{n}{k}, & k\mid n,\\
\varnothing, & \text{otherwise},
\end{cases}
\qquad 1\le n\le N_{\max},\ 1\le k\le K.
\end{equation}
Its binary support is $B(n,k)=\1_{\{k\mid n\}}$.

\subsection{Rays and angles}
For a fixed quotient $q\in\N$, the set of lattice points with $T(n,k)=q$ satisfies $n=qk$, i.e.\ a ray from the origin with slope $q$. Define the ray angle
\begin{equation}
\theta(q)=\arctan(q)\in (0,\pi/2).
\end{equation}

\subsection{Primitive cells}
The usual lattice visibility $\gcd(n,k)=1$ trivializes under $k\mid n$. Instead we define \emph{primitive factor cells} by
\begin{equation}
B^{\mathrm{prim}}(n,k)=\1_{\{k\mid n,\ \gcd(k,n/k)=1\}}.
\end{equation}
This retains only factor pairs with no shared prime factors.

\section{Diagonal ray algebra around a node $N$}
Fix an integer $N$ (the ``node''). Consider the diagonals $n+k=N$ and $n-k=N$.

\begin{proposition}[Diagonal characterization]
Let $k\in\N$. Then the cell on the diagonal $n+k=N$ is non-empty if and only if $k\mid N$, and in that case
\[
T(N-k,k)=\frac{N}{k}-1.
\]
Similarly, the cell on the diagonal $n-k=N$ is non-empty if and only if $k\mid N$, and in that case
\[
T(N+k,k)=\frac{N}{k}+1.
\]
\end{proposition}

Define the diagonal spectra
\begin{equation}
S^-(N)=\{d-1:\ d\mid N\},\qquad S^+(N)=\{d+1:\ d\mid N\}.
\end{equation}
Primes observed on the diagonals correspond to divisibility constraints: $p\in S^-(N)$ iff $(p+1)\mid N$; $p\in S^+(N)$ iff $(p-1)\mid N$.

\section{Primitive Geometric Table Graph}
\subsection{Windowed bipartite graph}
Fix a center $N_0$, window half-height $h$, and width $K$. Let
\[
I=[N_0-h,N_0+h]\cap \N.
\]
Define a bipartite graph $G=(U\sqcup V,E,w)$:
\begin{align}
U&=I \quad \text{(rows / integers)},\\
V&=\{q\in\N:\ \exists n\in I,\ \exists k\le K \text{ with } n=kq\},\\
E&=\{(n,q): n\in I,\ \exists k\le K: n=kq,\ \gcd(k,q)=1\}.
\end{align}
Weights can be chosen as $w(n,q)=1$ (unweighted), or geometrically by $w(n,q)=\theta(q)=\arctan(q)$, or scale-weighted by $w(n,q)=\log q$.

\subsection{Adjacency, Laplacian, and spectrum}
Let $A$ be the (weighted) adjacency matrix on $U\cup V$, $D$ its degree matrix, and $L=D-A$ the Laplacian. We typically compare normalized Laplacians
\[
L_{\mathrm{norm}}=I-D^{-1/2}AD^{-1/2}.
\]
We define the \emph{spectral fingerprint} of a window by a vector of low-frequency eigenvalues
\[
\bm{\lambda}_r(N_0;h,K)=\bigl(\lambda_1,\ldots,\lambda_r\bigr),
\]
together with summary features (spectral gap, entropy, localization measures).

\section{Snowflakes and ``full rays''}
\subsection{Partial snowflakes via sieve constraints}
For $K\ge 2$, define the \emph{partial snowflake condition}
\begin{equation}
FS_K(N):\quad \forall p\le K\ (p \text{ prime}),\quad N\not\equiv \pm 1 \pmod p.
\end{equation}
This is equivalent to requiring $p\nmid (N-1)$ and $p\nmid(N+1)$ for all primes $p\le K$.

\begin{theorem}[CRT infinitude of partial snowflakes]
For each fixed $K\ge 2$, the set of $N$ satisfying $FS_K(N)$ is infinite and is a union of residue classes modulo $Q_K=\prod_{p\le K}p$.
\end{theorem}

\subsection{Full snowflakes (twin-prime centers)}
Define \emph{full snowflakes} by the scale-coupled condition
\[
FS_\ast(N):\quad \forall p\le \sqrt{N+1}\ (p \text{ prime}),\quad N\not\equiv \pm 1 \pmod p.
\]
\begin{theorem}[Full snowflakes $\Leftrightarrow$ twin primes]
For even $N$, $FS_\ast(N)$ holds if and only if $N-1$ and $N+1$ are prime.
\end{theorem}

\section{Experimental protocol and reproducibility}
\subsection{Default parameters}
Default experimental parameters:
\[
h=200,\qquad K=200,
\]
and two weight modes: $w\equiv 1$ and $w(n,q)=\arctan(q)$.

\subsection{Outputs (Evidence Pack)}
Each run produces:
\begin{itemize}[leftmargin=2em]
  \item \texttt{params.json} (window, weights, filters);
  \item \texttt{nodes.json} (vertex maps);
  \item \texttt{edges.csv} (edge list);
  \item \texttt{eigenvalues.json} and \texttt{metrics.json};
  \item \texttt{checksums.sha256} over all artifacts.
\end{itemize}

\section{Discussion and open problems}
We propose studying whether spectral features reliably discriminate: (i) highly composite nodes, (ii) partial snowflakes $FS_K$ with large $K$, and (iii) full snowflakes (twin-prime centers). Proving infinitely many full snowflakes is equivalent to the Twin Prime Conjecture; nonetheless, the Geometric Table framework may expose new invariants and heuristics for ``ray stability'' and factor-geometry.

\section{Empirical Results}
\subsection{Setup}
We use windows with $h=200$, $K=200$, primitive filtering, and a fixed RNG seed.
The core window uses $r=30$ around the center. We report results for $w\equiv 1$
and the IDF weight. Centers are sampled uniformly from $[500,1500]$.

\subsection{Giant Component Spectra}
Normalized Laplacian spectra on the full graph contain many zeros, matching the
number of connected components. We therefore compute spectral summaries on the
giant component (GC), where $\lambda_2$ and entropy reflect structure rather than
disconnectedness.

\subsection{Twin-Center Effect in the Core Window}
We compare twin-centers (both $N\pm 1$ prime) against non-twins using core-window
GC metrics. The group summaries are in Table~\ref{tab:group-stats}.

\begin{table}[h]
\centering
\input{out/latex_table_groups_ones.tex}
\caption{Group statistics for core-window GC metrics (weight: ones).}
\label{tab:group-stats}
\end{table}

\subsection{Controlling for Composite Density}
We regress core GC gap against twin label, core edges, and GC fraction to control
for composite-density confounds. Figure~\ref{fig:core-gap-scatter} shows the
core-gap vs core-edges relationship, with highlighted centers.

\begin{figure}[h]
\centering
\includegraphics[width=0.75\linewidth]{fig/core_gap_scatter_edges.png}
\caption{Core gap vs core edges (twins vs non-twins; highlighted centers).}
\label{fig:core-gap-scatter}
\end{figure}

\begin{figure}[h]
\centering
\includegraphics[width=0.7\linewidth]{fig/core_gap_box_twins.png}
\caption{Core GC gap: twins vs non-twins.}
\label{fig:core-gap-box}
\end{figure}

\begin{figure}[h]
\centering
\includegraphics[width=0.7\linewidth]{fig/core_entropy_box_twins.png}
\caption{Core GC entropy: twins vs non-twins.}
\label{fig:core-entropy-box}
\end{figure}

\begin{figure}[h]
\centering
\includegraphics[width=0.7\linewidth]{fig/core_gap_hist.png}
\caption{Core GC gap distribution with highlighted centers.}
\label{fig:core-gap-hist}
\end{figure}

\paragraph{Disclaimer.}
These results are empirical; they do not prove infinitude of twin primes nor
circumvent known theoretical barriers. They provide evidence of spectral patterns
consistent with the proposed hypotheses.

\appendix
\section{Reference algorithm (pseudocode)}
\begin{algorithm}[H]
\caption{Build primitive bipartite graph from a window}
\begin{algorithmic}[1]
\Require center $N_0$, half-height $h$, width $K$
\State $I \gets \{N_0-h,\dots,N_0+h\}\cap\N$
\State $U \gets I$, $V\gets \emptyset$, $E\gets \emptyset$
\For{$n \in I$}
  \For{$k \in \{1,\dots,K\}$}
    \If{$k \mid n$}
      \State $q \gets n/k$
      \If{$\gcd(k,q)=1$}
        \State add $q$ to $V$
        \State add edge $(n,q)$ to $E$
      \EndIf
    \EndIf
  \EndFor
\EndFor
\State \Return $(U,V,E)$
\end{algorithmic}
\end{algorithm}

\bibliographystyle{plain}
% \bibliography{references}
\end{document}
